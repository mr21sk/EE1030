%iffalse
\let\negmedspace\undefined
\let\negthickspace\undefined
\documentclass[journal,12pt,twocolumn]{IEEEtran}
\usepackage{cite}
\usepackage{amsmath,amssymb,amsfonts,amsthm}
\usepackage{algorithmic}
\usepackage{graphicx}
\usepackage{textcomp}
\usepackage{xcolor}
\usepackage{txfonts}
\usepackage{listings}
\usepackage{enumitem}
\usepackage{mathtools}
\usepackage{gensymb}
\usepackage{comment}
\usepackage[breaklinks=true]{hyperref}
\usepackage{tkz-euclide} 
\usepackage{listings}
\usepackage{gvv}                                        
%\def\inputGnumericTable{}                                 
\usepackage[latin1]{inputenc}                                
\usepackage{color}                                            
\usepackage{array}                                            
\usepackage{longtable}                                       
\usepackage{calc}                                             
\usepackage{multirow}                                         
\usepackage{hhline}                                           
\usepackage{ifthen}                                           
\usepackage{lscape}
\usepackage{tabularx}
\usepackage{array}
\usepackage{float}
\usepackage{multicol}
\newcounter {sectioicolsn}


\newtheorem{theorem}{Theorem}[section]
\newtheorem{problem}{Problem}
\newtheorem{proposition}{Proposition}[section]
\newtheorem{lemma}{Lemma}[section]
\newtheorem{corollary}[theorem]{Corollary}
\newtheorem{example}{Example}[section]
\newtheorem{definition}[problem]{Definition}
\newcommand{\BEQA}{\begin{eqnarray}}
\newcommand{\EEQA}{\end{eqnarray}}
\newcommand{\define}{\stackrel{\triangle}{=}}
\theoremstyle{remark}
\newtheorem{rem}{Remark}

% Marks the beginning of the document

\begin{document}
\bibliographystyle{IEEEtran}
\vspace{3cm}

\title{Assignment-01}
\author{EE24BTECH11036 - KRISHNA PATIL}
\maketitle
\newpage
\bigskip

\renewcommand{\thefigure}{\theenumi}
\renewcommand{\thetable}{\theenumi}
\textbf{SECTION B : JEE MAINS / AIEEE}
\hfill
\begin{enumerate}
\item Two common tangents to the circle {$ x^2+y^2=2a^2 $} and parabola {$ y^2 = 8ax $} are \hfill{{$\brak{2002} $}}
\begin{enumerate}
\begin{multicols}{2}
\item  {$ x = \pm { \brak{y+2a}} $}
\item  {$ y = \pm { \brak{x+2a}} $}
\item  {$ x = \pm { \brak{y+a}} $}
\item  {$ y = \pm { \brak{x+a}} $}
\end{multicols}
\end{enumerate}
\hfill
\item The normal at a point {$ \brak{bt_1^2 ,2bt_1} $} on a parabola meets the parabola again in the point {$ \brak{bt_2^2 ,2bt _2} $}  ,then \hfill {{$ \brak{2003} $}}
\begin{enumerate}
\begin{multicols}{2}
\item  {$ t_2 = t_1 + \frac{2}{t_1} $}
\item  {$ t_2 = -t_1 - \frac{2}{t_1} $}
\item  {$ t_2 = -t_1 + \frac{2}{t_1} $}
\item  {$ t_2 = t_1 - \frac{2}{t_1} $}
\end{multicols}
\end{enumerate}
\hfill
\item The foci of the ellipse {$ \frac{x^2}{16} + \frac{y^2}{b} $} and the hyperbola {$ \frac{x^2}{144} - \frac{y^2}{81} = \frac{1}{25} $} coincide. Then the value of {$ b^2 $} is 
\hfill{{$ \brak{2003} $}}
\begin{enumerate}
\begin{multicols}{2}
\item 9	
\item 1
\item 5
\item 7
\end{multicols}
\end{enumerate}
\hfill
\item If a $ \neq 0 $ and the line {$ 2bx +3cy + 4d = 0 $} passes through the points of intersection of the parabolas {$ y^2 = 4ax $} and {$  x^2 = 4ay $} , then 
\hfill
\hfill{{$\brak {2004} $}}
\begin{enumerate}
\begin{multicols}{2}
\item {$ d^2 + \brak{3b-2c}^2 = 0 $}
\item {$ d^2 + \brak{3b+2c}^2 = 0 $}
\item {$ d^2 + \brak{2b-3c}^2 = 0 $}
\item {$ d^2 + \brak{2b+3c}^2 = 0 $}
\end{multicols}
\end{enumerate}
\hfill
\item The eccentricity of an ellipse, with its centre at the origin, is {$ \frac{1}{2} $} {$ \cdot $} If one of the directrices is {$ x = 4 $}, then the equation of the ellipse is: \hfill{{$ \brak{2004} $}}
\begin{enumerate}
\begin{multicols}{2}
\item {$ 4x^2+3y^2 = 1 $}
\item {$ 3x^2+4y^2 = 12 $}
\item {$ 4x^2+3y^2 = 12 $}
\item {$ 3x^2+4y^2 = 1 $}
\end{multicols}
\end{enumerate}
\hfill 
\item Let {$\vec{P}$} be the point {$ \brak{1,0} $} and {$\vec{Q}$} a point on the locus {$ y^2 = 8x $} . The locus of mid point of {$ PQ $} is \hfill{{$ \brak{2005} $}}
\begin{enumerate}
\begin{multicols}{2}
\item{$ y^2-4x=2 $}
\item{$ y^2+4x=2 $}
\item{$ x^2+4y=2 $}
\item{$ x^2-4y=2 $}
\end{multicols}
\end{enumerate}
\hfill
\item The locus of a point {$\vec{P}$} {$ \brak{\alpha,\beta} $} moving under the condition that the line {$ y = \alpha x + \beta $} is a tangent to the hyperbola {$ \frac{x^2}{a^2} - \frac{y^2}{b^2} $} is \hfill{{$ \brak{2005} $}}
\begin{enumerate}
\begin{multicols}{2}
\item an ellipse
\item a circle
\item a parabola
\item a hyperbola
\end{multicols}
\end{enumerate}
\hfill
\item An ellipse has {$ OB $} as semi minor axis, {$\vec{F}$} and {$ \vec{F'} $} focii and the angle {$ FBF' $} is a right angle {$ \cdot $} Then the eccentricity of the ellipse is
\hfill
\hfill{{$ \brak{2005} $}}
\begin{enumerate}
\begin{multicols}{2}
\item {$ \frac{1}{\sqrt2} $}
\item {$ \frac{1}{2} $}
\item {$ \frac{1}{4} $}
\item {$ \frac{1}{\sqrt3} $}
\end{multicols}
\end{enumerate}
\hfill
\item The locus of the vertices of the family of parabolas {$ y = \frac{a^3 x^2}{3} + \frac{a^2 x}{2} - 2a  $} is 
\hfill{{$ \brak{2006} $}}
\begin{enumerate}	
\begin{multicols}{2}
\item {$ xy = \frac{105}{64} $}
\item {$ xy = \frac{3}{4} $}
\item {$ xy = \frac{35}{16} $}
\item {$ xy = \frac{64}{104} $}
\end{multicols}
\end{enumerate}
\hfill
\item In an ellipse, the distance between its foci is 6 and minor axis is 8. Then it's eccentricity is
\hfill
\hfill{{$ \brak{2006} $}}
\begin{enumerate}
\begin{multicols}{2}
\item {$ \frac{3}{5} $}
\item {$ \frac{1}{2} $}
\item {$ \frac{4}{5} $}
\item {$ \frac{1}{\sqrt5} $}
\end{multicols}
\end{enumerate}
\hfill
\item Angle between the tangents to the curve {$ y = x^2 -5x + 6 $} at the points {$ \brak{2,0} $} and {$ \brak{3,0} $} is
\hfill
\hfill{{$ \brak{2006} $}}
\begin{enumerate}
\begin{multicols}{2}
\item {$ {\pi} $}
\item {$ \frac{\pi}{2} $}
\item {$ \frac{\pi}{6} $}
\item {$ \frac{\pi}{4} $}
\end{multicols}
\end{enumerate}
\hfill
\item For the Hyperbola {$ \frac{x^2}{cos^2 \alpha} - \frac{y^2}{sin^2 \alpha} = 1 $} , which of the  following remains constant when {$ \alpha $} varies = ? 
\hfill
\hfill{{$ \brak{2007} $}}
\begin{enumerate}
\item abscissae of vertices 
\item abscissae of foci
\item eccentricity
\item directrix
\end{enumerate}
\hfill
\item The equation of a tangent to the parabola {$ y^2 = 8x $} is {$ y = x + 2 $} {$ \cdot $}  The point on this line from which the other tangent to the parabola is perpendicular to the given tangent is
\hfill 
\hfill{{$\brak{2007}$}}
\begin{enumerate}
\begin{multicols}{2}
\item {$ \brak{2,4} $}
\item {$ \brak{-2,0} $}
\item {$ \brak{-1,1} $}
\item {$ \brak{0,2} $}
\end{multicols}
\end{enumerate}
\hfill
\item The normal to a curve at {$\vec{P}$}{$ \brak{x, y} $} meets the x-axis at {$\vec{G}$} {$ \cdot $} If the distance of G from the origin is twice the abscissa of {$\vec{P}$} , then the curve is \hfill {{$ \brak{2007} $}}
\begin{enumerate}
\begin{multicols}{2}
\item circle
\item hyperbola
\item ellipse
\item parabola
\end{multicols}
\end{enumerate}
\hfill
\item  A focus of an ellipse is at the origin {$ \cdot $} The directrix is the line {$ x = 4 $} and the eccentricity is {$ \frac{1}{2} $} {$ \cdot $} Then the length of the semi-major axis is 
\hfill
\hfill { $ \brak{2008} $}
\begin{enumerate}
\begin{multicols}{2}
\item {$ \frac{8}{3} $}
\item {$ \frac{2}{3} $}
\item {$ \frac{4}{3} $}
\item {$ \frac{5}{3} $}
\end{multicols}	
\end{enumerate}
\hfill
\end{enumerate}
\end{document}
