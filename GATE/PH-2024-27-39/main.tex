\let\negmedspace\undefined
\let\negthickspace\undefined
\documentclass[journal]{IEEEtran}
\usepackage[a5paper, margin=10mm, onecolumn]{geometry}
\usepackage{lmodern} % Ensure lmodern is loaded for pdflatex
\usepackage{tfrupee} % Include tfrupee package

\setlength{\headheight}{1cm} % Set the height of the header box
\setlength{\headsep}{0mm}     % Set the distance between the header box and the top of the text

\usepackage{gvv-book}
\usepackage{gvv}
\usepackage{cite}
\usepackage{amsmath,amssymb,amsfonts,amsthm}
\usepackage{algorithmic}
\usepackage{graphicx}
\usepackage{textcomp}
\usepackage{xcolor}
\usepackage{txfonts}
\usepackage{listings}
\usepackage{enumitem}
\usepackage{mathtools}
\usepackage{gensymb}
\usepackage{comment}
\usepackage[breaklinks=true]{hyperref}
\usepackage{tkz-euclide} 
\usepackage{listings}
\usepackage{gvv}                                        
\def\inputGnumericTable{}                                 
\usepackage[latin1]{inputenc}                                
\usepackage{color}                                            
\usepackage{array}                                            
\usepackage{longtable}                                       
\usepackage{calc}                                             
\usepackage{multirow}                                         
\usepackage{hhline}                                           
\usepackage{ifthen}                                           
\usepackage{lscape}
\begin{document}

\bibliographystyle{IEEEtran}
\vspace{3cm}

\title{PH-2024-27-39}
\author{EE24BTECH11036 - Krishna Patil}
% \maketitle
% \newpage
% \bigskip
{\let\newpage\relax\maketitle}
\begin{enumerate}
\setcounter{enumi}{26}
\item The temperature dependence of the electrical conductivity $\brak{\sigma}$ of three intrinsic semiconductors $ A$ ,$B$ and $C$ is shown in the figure.
\begin{figure}[!ht]
\centering
\resizebox{0.6\textwidth}{!}{
\begin{circuitikz}
\tikzstyle{every node}=[font=\normalsize]
\draw [->, >=Stealth] (7.75,11.25) -- (7.75,17);
\draw [->, >=Stealth] (7,12) -- (14.75,12);
\draw [short] (8.5,15.25) -- (10,12.75);
\draw [short] (9.25,15.25) -- (11.5,12.75);
\draw [short] (10.25,15.25) -- (13,13.5);
\node [font=\normalsize] at (9.5,12.75) {A};
\node [font=\normalsize] at (11,12.75) {C};
\node [font=\normalsize] at (12.75,13.25) {B};
\node [font=\normalsize] at (6.75,14.5) {$\ln\brak{\sigma}$};
\node [font=\normalsize] at (14.25,11.5) {$T^{-1}$};
\end{circuitikz}
}
\label{fig:my_label}
\end{figure}
Let $E_a$,$E_b$ and $E_c$ be the bandgaps of $A$,$B$ and $C$ respectively . Which one of the following relations are correct ? 
\begin{enumerate}
\item $E_c>E_a>E_b$
\item $E_b>E_c>E_a$ 
\item $E_a>E_b>E_c$
\item $E_a>E_c>E_b$
\end{enumerate}
\item Following trial waveforms \begin{center} $\phi_1 =e^{Z^{\prime}\brak{r_1+r_2}}$ \end{center}  and \begin{center} $\phi_2 =e^{Z^{\prime}\brak{r_1+r_2}}\brak{1+g\abs{\overrightarrow{r_1}-\overrightarrow{r_2}}}$ \end{center} are used to get a variational estimate of the ground state energy of helium atom. $Z^{\prime}$ and g are variational parameters , $\overrightarrow{r_1}$ and $\overrightarrow{r_2}$ are position vectors of the electrons. Let $E_0$ be the exact ground state energy of helium atom. $E_1$ and $E_2$ are the variational estimates of the ground state energy of the helium atom corresponding to $\phi_1$ and $\phi_2$ respectively .Which one of the following options is true ?
\begin{enumerate}
\item $E_1 \leq E_0 , E_2 \leq E_0 , E_1 \geq E_2$ 
\item $E_1 \geq E_0 , E_2 \leq E_0 , E_1 \geq E_2$ 
\item $E_1 \leq E_0 , E_2 \geq E_0 , E_1 \leq E_2 $
\item $E_1 \geq E_0 , E_2 \geq E_0 , E_1 \geq E_2 $
\end{enumerate} 
\newpage
\item The wave function for a particle is given by the form  $e^{\brak{i \alpha x + \beta t}}$ , where  $\alpha$  and  $\beta$ are real constants. In which one of the following potentials $V(x)$ , the particle is moving?
\begin{enumerate}
\begin{multicols}{2}
\item $V\brak{x} \propto \alpha x^2 $
\item $V\brak{x} \propto e^{\brak{-\alpha x}} $
\item $V\brak{x} = 0 $
\item $V\brak{x} \propto \sin\alpha x$ 
\end{multicols}
\end{enumerate}
\item  Consider a volume integral
\begin{center}
$I = \int_V v^2 \left( \frac{1}{r} \right) dV$
\end{center}
over a volume $V$, where $r = \sqrt{x^2 + y^2 + z^2}$. Which of the following statements is/are correct?
\begin{enumerate}
\item $I = -4\pi$ , if $r = 0$ is inside the volume $V$ 
\item Integrand vanishes for $r \neq 0$ 
\item $I = 0$, if $r = 0$ is not inside the volume $V$ 
\item Integrand diverges as $r \to \infty$ 
\end{enumerate}
\item The complex function $e^{-\frac{2}{z - 1}}$ has
\begin{enumerate}
\begin{multicols}{2}
    \item[(A)] a simple pole at \( z = 1 \)
    \item[(B)] an essential singularity at \( z = 1 \)
    \item[(C)] a residue equal to \( -2 \) at \( z = 1 \)
    \item[(D)] a branch point at \( z = 1 \)
\end{multicols} 
\end{enumerate}
\item The minimum number of basic logic gates required to realize the Boolean expression $B \cdot\brak{A+B}+A\cdot\brak{\overline{B}+A}$  is \underline{\hspace{2cm}} (in integer).\\
\item The vapor pressure $\brak{P}$ of solid ammonia is given by $\ln{\brak{p}}=23.03-\frac{3754}{T}$,while that of liquid ammonia is given by $\ln{\brak{p}}=29.49-\frac{3063}{T}$,where $T$ is the temperature in $K$.The temperature of the triple point of ammonia is \rule{2cm}{0.4pt} $K$ $\brak{\text{rounded off to two decimal places}}$. \\
\item The electric field in a region depends only on $x$ and $y$ coordinates as
\begin{center}
 $E = k\frac{x\hat{x}+y\hat{y}}{x^2 + y^2}$   
\end{center}
where $k$ is a constant. The flux of $E$ through the surface of a sphere of radius $R$ with its center at the origin is $n \pi k R$, where the value of $n$ is \underline{\hspace{1cm}} (in integer).\\
\item The Hamiltonian of a system of $N$ particles in volume $V$ at temperature $T$ is
\begin{center}
$H = \sum\limits_{i=1}^{2N} a_i q_i^2 + \sum\limits_{i=1}^{2N} b_i p_i^2$
\end{center}
where $a_i$ and $b_i$ are positive constants. The ensemble average of the Hamiltonian is $\alpha N k_B T$, where $k_B$ is the Boltzmann constant. The value of $\alpha$ is \underline{\hspace{1cm}} (in integer).
\newpage
\item Binding energy and rest mass energy of a two-nucleon bound state are denoted by $ B$  and  $mc^2$, respectively, where $c$ is the speed of light. The minimum energy of a photon required to dissociate the bound state is
\begin{enumerate}
\item B
\item $B\brak{ 1 + \frac{B}{2mc^2}}$
\item $B\brak{1 - \frac{B}{2mc^2}}$
\item $B-mc^2$
\end{enumerate}
\item The spin-orbit interaction in a hydrogen-like atom is given by the Hamiltonian
\begin{center}
 $H^{\prime} = -k\overrightarrow{L}\cdot\overrightarrow{S} $  
\end{center}
where $k$  is a real constant. The splitting between levels $^2p_{\frac{3}{2}}$ and $^2p_{\frac{1}{2}}$ due to this interaction is:
\begin{enumerate}
	\item $\frac{1}{2}k{\hbar}^2$ 
	\item $\frac{3}{2}k{\hbar}^2$ 
	\item $\frac{3}{4}k{\hbar^2}^2$ 
	\item $2k{\hbar}^2$
\end{enumerate}
\item Consider the Lagrangian $L=m\dot{x}\dot{y}-m\omega_0^2xy$. If $p_x$ and $p_y$ denote the generalized momenta conjugate to $x$ and $y$, respectively, then the canonical equations of motion are:
\begin{enumerate}
\item $\dot{x}=\frac{p_x}{m},\dot{p_x}=-m\omega_0^2x,\dot{y}=\frac{p_y}{m},\dot{p_y}=-m\omega_0^2y$
\item $\dot{x} = \frac{p_x}{m},\dot{p_x}=m\omega_0^2x,\dot{y}=\frac{p_y}{m},\dot{p_y}=m\omega_0^2y$
\item $\dot{x}=\frac{p_y}{m},\dot{p_x}=-m\omega_0^2 y,\dot{y}=\frac{p_x}{m},\dot{p_y}=-m\omega_0^2x$
\item $\dot{x}=\frac{p_y}{m},\dot{p_x}=m\omega_0^2y,\dot{y}=\frac{p_x}{m},\dot{p_y}=m\omega_0^2 x$
\end{enumerate}
\item The X-ray diffraction pattern of a monoatomic cubic crystal with rigid spherical atoms of radius $1.56A^{\circ}$ shows several Bragg reflections of which the reflection appearing at the lowest $2\theta$ value is from \brak{111} plane .If the wavelength of X-ray used is $0.78A^{\circ}$ ,the Bragg angle $\brak{in 2\theta , \text{rounding off to one decimal place}}$ corresponding to this reflection and the crystal structure ,respectively,are
\begin{enumerate}
\item $21.6^{\circ}$ and Body centered cubic
\item $17.6^{\circ}$ and face centered cubic
\item $10.8^{\circ}$ and Body centered cubic
\item $8.8^{\circ}$ and face centered cubic
\end{enumerate}
\end{enumerate}
\end{document}      
