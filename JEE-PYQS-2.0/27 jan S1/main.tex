%iffalse
\let\negmedspace\undefined
\let\negthickspace\undefined
\documentclass[journal,12pt,onecolumn]{IEEEtran}
\usepackage{cite}
\usepackage{amsmath,amssymb,amsfonts,amsthm}
\usepackage{algorithmic}
\usepackage{graphicx}
\usepackage{textcomp}
\usepackage{xcolor}
\usepackage{txfonts}
\usepackage{listings}
\usepackage{enumitem}
\usepackage{mathtools}
\usepackage{gensymb}
\usepackage{comment}
\usepackage[breaklinks=true]{hyperref}
\usepackage{tkz-euclide} 
\usepackage{listings}
\usepackage{gvv}                                        
%\def\inputGnumericTable{}                                 
\usepackage[latin1]{inputenc}     
\usepackage{xparse}
\usepackage{color}                                            
\usepackage{array}                                            
\usepackage{longtable}                                       
\usepackage{calc}                                             
\usepackage{multirow}
\usepackage{multicol}
\usepackage{hhline}                                           
\usepackage{ifthen}                                           
\usepackage{lscape}
\usepackage{tabularx}
\usepackage{array}
\usepackage{float}
\newtheorem{theorem}{Theorem}[section]
\newtheorem{problem}{Problem}
\newtheorem{proposition}{Proposition}[section]
\newtheorem{lemma}{Lemma}[section]
\newtheorem{corollary}[theorem]{Corollary}
\newtheorem{example}{Example}[section]
\newtheorem{definition}[problem]{Definition}
\newcommand{\BEQA}{\begin{eqnarray}}
\newcommand{\EEQA}{\end{eqnarray}}
\newcommand{\define}{\stackrel{\triangle}{=}}
\theoremstyle{remark}
\newtheorem{rem}{Remark}
% Marks the beginning of the document
\begin{document}
\title{Assignment 1}
\author{EE24Btech11036 - Krishna Hanumanth Patil}
\maketitle
\renewcommand{\thefigure}{\theenumi}
\renewcommand{\thetable}{\theenumi}
\begin{enumerate}

\item If f(x) = $ \myvec{cosx & -sinx & 0 \\ sinx & cosx & 0 \\ 0 & 0 & 1 } $ then \\ Statement 1 : f(-x) is inverse of f(x) . \\ Statement 2 :f(x + y) = f(x)f(y) . 

\hfill{\brak{\text{Jan 2024}}}
\begin{enumerate}
\item Both are true 
\item Both are false 
\item Only statement 1 is true 
\item Only statement 2 is true 
\end{enumerate}

\\ 

\item If $S=\cbrak{z\in\mathbb{C}:\abs{z-i}=\abs{z-1}=\abs{z+i}}$ , then the n(S) is :

\hfill{\brak{\text{Jan 2024}}}
\begin{enumerate}
\begin{multicols}{4}
\item $ 2 $
\item $ 3 $
\item $ 1 $
\item $ 0 $
\end{multicols}
\end{enumerate}

\item If a = $ \lim_{x\to0}\frac{\brak{\sqrt{1+\sqrt{1+x^4}}}-\brak{\sqrt{2}}}{x^4} $ and  b = $ \lim_{x\to0}\frac{sin^2x}{\sqrt{2}-\sqrt{1+cosx}} $ , find $ ab^3 $

\hfill{\brak{\text{Jan 2024}}}
\begin{enumerate}
\begin{multicols}{4}
\item $ 36 $
\item $ 32 $
\item $ 25 $
\item $ 30 $
\end{multicols}
\end{enumerate}

\item Let $ \int_0^1 \frac{1}{\sqrt{x+1}+\sqrt{x+3}} \, dx $ = a+b$ \sqrt{2} $+$ c\sqrt{3} $ where a,b,c are rational numbers, then  $ 2a+3b-4c $ is equal to :
        
\hfill{\brak{\text{Jan 2024}}}  
\begin{enumerate}
\begin{multicols}{4}
\item $ 10 $
\item $ 8 $
\item $ 4 $
\item $ 7 $
\end{multicols}
\end{enumerate}

\item The distance of the point \brak{7,-2,11} from the line $ \frac{x-6}{1}=\frac{y-4}{0}=\frac{z-8}{3} $ along the line $ \frac{x-5}{2}=\frac{y-4}{-3}=\frac{z-5}{6} $ , is :

\hfill{\brak{\text{Jan 2024}}}
\begin{enumerate}   
\begin{multicols}{4}
\item $14$
\item $21$
\item $12$
\item $18$                                                                        
\end{multicols}
\end{enumerate}

\item The length of chord of thw ellipse $ \frac{x^2}{25} + \frac{y^2}{16} = 1 $ ,whose midpoint is $ \brak{1,\frac{2}{5}} $ , is equal to :

\hfill{\brak{\text{Jan 2024}}}
\begin{enumerate}
\begin{multicols}{4}
\item $ \frac{\sqrt{1691}}{5} $
\item $ \frac{\sqrt{2009}}{5} $
\item $ \frac{\sqrt{1741}}{5} $
\item $ \frac{\sqrt{1541}}{5} $
\end{multicols}
\end{enumerate}

\item Find number of common terms in the two given series ; \\ $ 4, 9, 14, 19,... $ up to $ 25 $ terms and $ 3, 9, 15, 21,... $ up to $ 37 $ terms

\hfill{\brak{\text{Jan 2024}}}
\begin{enumerate}
\begin{multicols}{4}
\item $9$
\item $8$
\item $5$
\item $7$
\end{multicols}
\end{enumerate}

\item If the shortest distance of the parabola $ y^2=4x $ from the centre of the circle $ x^2+y^2-4x-16y+64=0 $ is d ,then $ d^2 $ is equal to :

\hfill{\brak{\text{Jan 2024}}}  
\begin{enumerate}   
\begin{multicols}{4}
\item $36$
\item $20$
\item $16$
\item $24$
\end{multicols}	
\end{enumerate}

\item Let $S=\cbrak{1,2,3,...,10}$  . Suppose M is the set of all subsets of S , the relation  $R=\cbrak{ (A,B):A \cap B \neq \phi ;A,B \in M }$ is :

\hfill{\brak{\text{Jan 2024}}}
\begin{enumerate}
\item symmetric and transitive only 
\item symmetric only 
\item symmetric and reflexive only 
\item reflexive only 
\end{enumerate}

\item Let $x=x(t)$ and $y=y(t)$ be the solutions of the diffrential equations $\frac{dx}dt{}+ax=0$ and $\frac{dy}{dt}+by=0$ respectively, $a,b \in \mathbb{R}$ . Given that $x(0)=2$ , $y(0)=1$ and $3y(1)=2x(1)$ , the value of t , for which $x(t)=y(t)$ , is :  

\hfill{\brak{\text{Jan 2024}}}
\begin{enumerate}   
\begin{multicols}{2}
\item $ log_3 4 $
\item $ log_{\frac{4}{3}} 2 $
\item $ log_4 3 $
\item $ log_2 2 $
\end{multicols}
\end{enumerate}

\item If $\comb{n-1}{r} = (k^2 - 8)\comb{n}{r+1}$ , then the range of 'k' is 

\hfill{\brak{\text{Jan 2024}}}
\begin{enumerate}
\begin{multicols}{2}
\item $ k \in (2\sqrt{2},3] $
\item $ k \in \brak{2\sqrt{2},3} $
\item $ k \in [2,3) $
\item $ k \in \brak{2\sqrt{2},8} $
\end{multicols}
\end{enumerate}
\\

\item If the shortest distance between the lines $ \frac{x-4}{1}=\frac{y+1}{2}=\frac{z}{-3} $ and $ \frac{x-\lambda}{2}=\frac{y+1}{4}=\frac{z-2}{-5} $ , is $\frac{6}{\sqrt{5}}$ , then the sum of all possible values of $\lambda$ is :

\hfill{\brak{\text{Jan 2024}}}
\begin{enumerate}   
\begin{multicols}{4}
\item $10$
\item $5$
\item $8$
\item $7$                                                                        
\end{multicols}
\end{enumerate}

\item Let $\vec{a}=\hat{i}+2\hat{j}+\hat{k}$ , $\vec{b}=3(\hat{i}-\hat{j}+\hat{k})$ . Let $\vec{c}$ be the vector such that $\vec{a}$x$\vec{c}$=$\vec{b}$ and $\vec{a}$.$\vec{c}=3$ . Let $\vec{a}$.(($\vec{b}$x$\vec{c}$)-$\vec{b}$-$\vec{c}$) is equal to : 

\hfill{\brak{\text{Jan 2024}}}
\begin{enumerate}   
\begin{multicols}{4}
\item $24$
\item $36$ 
\item $32$ 
\item $20$                                                                         
\end{multicols}
\end{enumerate}

\item If A denotes the sum of all the coeeficients in the expansion of $(1-3x+10x^2)^{n}$ and B enotes the sum of all the coefficients in the expansion of $(1+x^2)^{n}$ , then :

\hfill{\brak{\text{Jan 2024}}}
\begin{enumerate}   
\begin{multicols}{4}   
\item $ A=B^3 $
\item $ A=3B $
\item $ B=A^3 $
\item $ 3A=B $
\end{multicols}
\end{enumerate}


\item Consider the line $L:4x+5y=20$ . Let two other lines are $L_1$ and $L_2$ which trisect the line L and pass through origin, then tangent of angle between lines $L_1$ and $L_2$ is

\hfill{\brak{\text{Jan 2024}}}
\begin{enumerate}
\begin{multicols}{4}
\item $ \frac{25}{41} $
\item $ \frac{30}{41} $
\item $ \frac{2}{5} $
\item $ \frac{3}{5} $
\end{multicols}
\end{enumerate}



\end{enumerate}
\end{document}
